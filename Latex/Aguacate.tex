\documentclass{article}
\usepackage[utf8]{inputenc}
\usepackage[spanish]{babel}

%% Sets page size and margins
\usepackage[a4paper,top=3cm,bottom=2cm,left=3cm,right=3cm,marginparwidth=1.75cm]{geometry}

%% Useful packages
\usepackage{listings}
\usepackage{caption}
\usepackage{amssymb,amsmath,amsthm}
\usepackage{mathtools} \usepackage{graphicx}
\usepackage[colorlinks=true, allcolors=blue]{hyperref}
\usepackage{float}
\usepackage{enumerate}
\usepackage{subcaption}
\usepackage{hyperref}
\newcommand{\Lagr}{\mathcal{L}}
\newcommand\norm[1]{\left\lVert#1\right\rVert}
\DeclareMathOperator*{\argmin}{arg\,min} 
\usepackage{tikz-cd}
\setlength{\parindent}{0.5pt}%
\newtheorem{theorem}{Teorema}
\newtheorem{definition}{Definición}
\newtheorem*{proposition}{Proposición}
\newtheorem*{corollary}{Corolario}
\newtheorem*{example}{Ejemplo}
\newtheorem*{lemma}{Lema}
\newtheorem{observation}{Observación}
\usepackage{fancyhdr}
\pagestyle{fancy}
\thispagestyle{empty}
\decimalpoint
\usepackage{setspace}
\usepackage{csvsimple}
\begin{document}
\onehalfspacing 

\title{Analisis del mercado de aguacates en Estados Unidos}

\author{José Pablo Ortiz, \\CIMAT}
\maketitle

\section{Introducción}
En este trabajo se va analizar el mercado de aguacates en Estados Unidos. En la primera parte de este trabajo se hará un análisis de los precios y montos de aguacates desde una perspectiva que no considere la relación de estas cantidades con el tiempo. Para esto se va a sintetizar la información importante en tablas, se realizaran pruebas de hipótesis para hacer comparaciones multiples, y se propondrá  rankings de la relación que guarda el precio con el tamaño, lugar de origen y mes del año. En la segunda parte se presentaran los resultados obtenidos de modelar la serie de tiempo del precio del aguacate utilizando la metodología ARIMA o Box-Jenkins. 
\section{Múltiples comparaciones}

En esta sección se busca encontrar la relación que existe entre el precio del aguacate y su tipo, variedad, tamaño, lugar de origen y mes del año. Para esto se utilizara pruebas de hipótesis. En los casos de tipo y variedad se empleara la prueba Mann-Whitney. Esto se debe a que estas variables solo tienen dos categorias: Hass y  \textit{variedades verdes} en el caso de variedad y orgánica y convencional en el caso de tipo. La prueba Mann-Whitney compara dos poblaciones $X$ y $Y$ y tiene como hipótesis nula que la probabilidad de que $X$ sea mayor que $Y$ es igual que la probabilidad de que $Y$ sea mayor $X$. La hipótesis alternativa es que la probabilidad  $X$ sea mayor que $Y$ es menor que la probabilidad de que $X$ sea menor que $Y$. Así, si se rechaza la hipótesis nula se estaría encontrando evidencia a favor de que $Y$ es mayor que $X$.
\\

\subsection*{¿La diferencia en precios es estadísticamente diferente para los diferentes tipos de fruta?}
El Cuadro 3 muestra las estadísticas descriptivas del tipo de fruta. La experiencia dicta que la fruta orgánica es más cara que la convencional. Es por esto que en este caso se utiliza la hipótesis alternativa de que el precio de la fruta orgánica es mayor que el de la convencional. El $p$-valor obtenido es de casi cero y, por lo tanto, se acepta este hecho.

\begin{center}
\begin{tabular}{|l|l|l|l|l|l|l|}
	\hline
		Tipo & Mínimo & Cuantil 0.25 & Cuantil 0.5 & Cuantil 0.75 & Máximo & Media \\ \hline
		Conventional & 5.5 & 21.5 & 30.25 & 41.75 & 90.25 & 32.55\\ \hline
 		Organic & 20.25 & 32.75 & 42.25 & 58.75 & 83.75 & 45.62\\ \hline
	\end{tabular}
	\captionof{table}{Estadísticas descriptivas del tipo de fruta.}
\end{center}
\subsection*{¿El precio  es estadísticamente diferente para diferentes variedades?}
Existen varios tipos de variedades de aguacate: Haas, Arad, Ettinger, fuerte, Pinkerton, Reed, entre otras. En las base de datos de los precios las variedades distintas a la Haas se consideran como una solo categoria: \textit{distinas variedades verde}. Las estadísticas descriptivas para la variedad Haas y las \textit{distinas variedades verdes} se muestran en el Cuadro 2. Esta información propone que en la prueba Mann Whitney se utilice como hipótesis nula que la variedad HAAS es tiene un costo más elevado que las \textit{distinas variedades verdes}. Al igual que en la pregunta anterior, se obtiene un $p$-valor cercano a cero y se rechaza la hipótesis nula. Luego se puede concluir que la fruta Hass es más cara que las \textit{variedades verdes} .

\begin{center}
\begin{tabular}{|l|l|l|l|l|l|l|}
	\hline
		Variedad & Minimo & Cuantil 0.25 & Cuantil 0.5 & Cuantil 0.75 & Maximo & Media \\ \hline
	Hass & 11.25 & 27.25 & 36.25 & 47.75 & 90.25 & 38.57\\ \hline
 \textit{Variedades verdes} & 5.5 & 13.0 & 17.5 & 24.0 & 51.5 & 18.60\\ \hline
	\end{tabular}
		\captionof{table}{Estadísticas descriptivas de la variedad de la fruta.}
\end{center}

Utilizar este mismo método para las relaciones faltantes no es posible debido a que cada variable tiene más de dos categorías. Este implica que se tienen que hacer uso de técnicas de comparación múltiple. Se utilizaran las pruebas Kruskal-Wallis y la de Dunn con ajuste de Bonferroni para hacer estas comparaciones. La prueba Kruskal-Wallis tiene como hipótesis nula que las diferentes poblaciones tienen una misma distribución, así al rechazarla significaría que sí existe un efecto entre la relación de las diferentes categorías y el precio. Luego, si se rechaza la hipótesis nula de Kruskal-Wallis se utiliza la prueba de Dunn con corrección de Bonferroni para hacer comparaciones entre todas las parejas . 

Se tienen dos objetivos en mente. El primero, buscar cuando no existe una diferencia significativa entre dos o más categorias y segundo, hacer un ranking de la relación que existe entre el precio y las diferentes categorías. 

\subsection*{¿El precio por tamaño es estadísticamente diferente ?}
Para contestar esta pregunta se utilizaran los datos únicamente de la fruta convencional mexicana. Los únicos tamaños de fruta convencional mexicana de los que se tienen datos son de 32, 36, 40, 48, 60, 70 y 84. Al observar el Cuadro 3 se nota que las estadísticas para los tamaños 32, 36, 40 y 48 son muy similares. Por el contrario, las estadísticas de los tamaños 60, 70 y 84 no comparten ninguna similaridad con los demás tamaños.

\begin{center}
\begin{tabular}{|l|l|l|l|l|l|l|}
\hline
		Tamaño & Minimo & Cuantil 0.25 & Cuantil 0.5 & Cuantil 0.75 & Maximo & Media \\ \hline
		 32s & 16.25 & 31.75 & 38.25 & 48.75 & 90.25 & 40.89\\ \hline
 36s & 16.75 & 32.25 & 38.25 & 48.75 & 90.25 & 40.93\\ \hline
 40s & 17.75 & 31.75 & 37.75 & 49.25 & 90.25 & 40.52\\ \hline
 48s & 20.25 & 30.75 & 37.25 & 48.25 & 90.25 & 40.40\\ \hline
 60s & 17.25 & 25.75 & 32.75 & 41.25 & 85.25 & 35.34\\ \hline
 70s & 15.25 & 22.875 & 27.75 & 34.75 & 81.75 & 30.56\\ \hline
 84s & 11.25 & 17.75 & 20.75 & 24.75 & 59.25 & 22.72\\ \hline
\end{tabular}
		\captionof{table}{Estadísticas descriptivas del tamaño de la fruta.}
\end{center}
Al realizar la prueba de Kruskal-Wallis  se obtiene un $p$-valor cercano a cero, por lo tanto, se rechaza la hipotesis nula. Luego, se puede utilizar la prueba de Dunn. Los resultados de esta prueba es que los tamaños 32, 36, 40 y 48 no presentan una diferencia significativa en el precio. Además, los tamaños 60, 70 y 84 son diferentes todos entre si. Luego al tomar la información del Cuadro 3 se hace el siguiente rankeo de los precios según el tamaño.

\begin{center}
	\begin{tabular}{|l|l|}
	\hline
		Grupo & Tamaños\\ \hline
		1 & 32s, 36s, 40s, 48s  \\ \hline
		2 & 60s \\ \hline
		3 & 70s \\ \hline
		4 & 84s \\ \hline
	\end{tabular}
    \captionof{table}{A table beside a figure}
\end{center}	
Estos resultados están en con la practica. La semilla de un aguacate es muy grande, entonces entre más pequeño sea un aguacate(que se de mayor de tamaño) menor cantidad de alimento tendra y  más barato será la fruta. Una cuestión interesante de este resultado es que los compradores son indiferente entre los aguacate de tamaño 32s, 36s, 40s, y 48s.

\subsection*{¿Es estadísticamente diferente el precio de la fruta por puerto de origen?}
En esta pregunta se utilizarán todos los datos. Los lugares de origen que se consideran para esta pregunta son: México, California, Peru, Chile y Republica Dominicana. El Cuadro 5 muestran las estadísticas relevantes para los lugares de origen. Como dato extra se menciona que México exporta alrededor del $76\%$ de los aguacates a Estados Unidos.
\begin{center}
\begin{tabular}{|l|l|l|l|l|l|l|l|l|}
\hline
		Tamaño & Minimo & Cuantil 0.25 & Cuantil 0.5 & Cuantil 0.75 & Maximo & Media & 10000lbs & \% \\ \hline
 Mexico& 11.25 & 25.75 & 34.25 & 43.25 & 90.25 & 36.37 & 1137760 & 0.766  \\ \hline
 Chile& 23.25 & 23.25 & 23.25 & 24.0 & 26.25 & 24.0 & 20615 & 0.013\\ \hline
 California & 12.25 & 33.25 & 42.25 & 54.25 & 83.75 & 43.36 & 187424 &0.126\\ \hline
Peru & 17.0 & 26.125 & 29.25 & 33.25 & 69.0 & 32.04 & 81640 & \\ \hline
 Florida & 5.5 & 10.5 & 14.5 & 21.0 & 51.5 & 16.26 &21434 & 0.014\\ \hline
 RD & 13.0 & 18.0 & 23.0 & 30.0 & 44.5 & 23.87& 34296&0.023\\ \hline
\end{tabular}
    \captionof{table}{A table beside a figure}
\end{center}

Se utiliza la misma metodología que en la pregunta anterior. El $p$-valor de la prueba Kruskal-Wallis es casi igual a 0. Los resultados de la prueba de Dunn nos indica que solo existe un empate -entre Chile y República Dominicana- y al tomar en cuenta el Cuadro 5 se propone el siguiente ranking.

\begin{center}
	\begin{tabular}{|l|l|}
	\hline
		Ranking & Origen\\ \hline
		1 & California\\ \hline
		2 & México\\ \hline
		3 & Peru \\ \hline
		3 & Chile, República Dominicana\\ \hline
		4 & Florida \\ \hline
	\end{tabular}
	  \captionof{table}{Ranking de los precios.}
\end{center}	
\subsection*{¿El precio promedio por mes es estadisticamente diferente? }
Ahora se va a estudiar la relación que tiene el precio con el mes. Para contestar esta pregunta se utilizaran los datos únicamente de la fruta convencional mexicana. Se decidió tomar como referencia el mes del año en lugar de la semana debido al número de comparaciones que se tienen que hacer. Para comparar los meses del año se tienen que realizar 66 comparaciones; mientras que si se tomaran  las semanas del año se tendrían que hacer 1326. El Cuadro 7 muestra la información importante. 
\begin{center}
\begin{tabular}{|l|l|l|l|l|l|l|l|l|}
\hline
		Mes & Minimo & Cuantil 0.25 & Cuantil 0.5 & Cuantil 0.75 & Maximo & Media & 10000lbs & \% \\ \hline
1 & 10.25 & 21.25 & 24.75 & 30.75 & 42.25 & 25.55 & 150235&0.101\\ \hline
 2 & 13.0 & 23.25 & 26.25 & 34.25 & 57.75 & 29.68&117700& 0.079\\ \hline
 3 & 12.25 & 26.25 & 32.75 & 44.25 & 67.75 & 35.40&130361&0.087\\ \hline
 4 & 11.25 & 29.75 & 39.25 & 49.25 & 65.25 & 39.64&129945&0.087\\ \hline
 5 & 13.25 & 33.25 & 39.25 & 48.75 & 72.75 & 40.75&126939&0.085\\ \hline
 6 & 8.0 & 28.75 & 39.25 & 48.25 & 85.25 & 40.36&134933&0.090\\ \hline
 7 & 5.5 & 23.5 & 34.25 & 46.75 & 83.75 & 36.08&134356&0.090\\ \hline
 8 & 5.5 & 22.0 & 34.25 & 52.25 & 82.75 & 36.80&122744&0.082\\ \hline
 9 & 7.5 & 21.0 & 28.25 & 41.25 & 90.25 & 33.61&118708&0.079\\ \hline
 10 & 8.5 & 18.0 & 26.0 & 33.25 & 68.75 & 27.48&122205&0.082\\ \hline
 11 & 8.0 & 14.5 & 18.25 & 27.75 & 59.75 & 21.25&90115&0.060\\ \hline
 12 & 8.5 & 16.75 & 19.0 & 27.25 & 44.75 & 22.11&106367&0.071\\ \hline
\end{tabular}
    \captionof{table}{Estadísticas descriptivas de los meses del año.}
\end{center}
Al utilizar la prueba de Kruskal-Wallis y después la prueba de Dunn se llega al siguiente ranking 

\begin{center}
	\begin{tabular}{|l|l|}
	\hline
		Ranking & Origen\\ \hline
		1 & 4,5,6\\ \hline
		2 & 3,7,8\\ \hline
		3 & 9\\ \hline
		4 & 2 \\ \hline
		5 & 1,10 \\ \hline
		6 & 11,12\\ \hline
	\end{tabular}
	  \captionof{table}{Ranking de los precios.}
\end{center}	

\section{Serie de Tiempo}
En la base de datos de los precios hay dos medidas de los precios \textit{Low Price} y \textit{High Price}. La frecuencia con la que se observan estas medidas es diaria. No obstante, para este análisis se tomaran los siguientes datos semanales: precio mínimo, precio promedio y  precio máximo. Se pretende hacer predicciones de estos 3 datos para cada tamaño significativamente diferente, ver Cuadro 4. Las predicciones solo se harán de la fruta convencional mexicana.

En los siguientes Cuadros se muestra el modelo ARIMA obtenido, las predicciones de 4 semanas en el futuro y los datos observados de esas 4 semanas.
\begin{center}
	\begin{tabular}{|l|l|l|l|}
	\hline
		Tamaño & Mínimo & Promedio& Máximo \\ \hline
		32s, 36s, 40s, 48s & (3,0,3)&(3,0,2) &(3,0,3\\ \hline
		60s & (2,0,3)& (2,0,3)&8(2,0,0)\\ \hline
		70s &(2,0,1) &(2,0,1) &(2,0,1) \\ \hline
		84s & (2,0,0) &(2,0,1) &(2,0,0j)\\ \hline
	\end{tabular}
	  \captionof{table}{Ranking de los precios.}
\end{center}	






\end{document}

